\documentclass[8pt]{extreport}

% monkey-patching tools
\usepackage{etoolbox}

% checks for xetex, luatex or either
\usepackage{ifxetex,ifluatex}
\newif\ifxelua
\ifxetex
	\xeluatrue
\else\ifluatex
	\xeluatrue
\else
	\xeluafalse
\fi\fi

% fonts, languages and encoding
% TODO make paths relative to the preamble, not the generated book
\ifxelua
	\usepackage{fontspec}
	\defaultfontfeatures{
		Ligatures = {TeX}
	}
	% assume that .fonts/ has been exported
	\setmainfont{PTF}[  % PT Serif Regular
		Path = ./fonts/PTSerif/,
		UprightFont = *55F,
		BoldFont = *75F,
		ItalicFont = *56F,
		BoldItalicFont = *76F
	]
	\setsansfont{PTS}[  % PT Sans Regular
		Path = ./fonts/PTSans/,
		UprightFont = *55F,
		BoldFont = *75F,
		ItalicFont = *56F,
		BoldItalicFont = *76F
	]
	\setmonofont{PTM}[  % PT Mono Regular
		Path = ./fonts/PTMono/,
		UprightFont = *55F,
		BoldFont = *75F,
	]
	\newfontfamily\familycaprm{PTZ}[  % PT Serif Caption Regular
		Path = ./fonts/PTSerif/,
		UprightFont = *55F,
		ItalicFont = *56F,
	]
	\newfontfamily\familycapsf{PTC}[  % PT Sans Caption Regular
		Path = ./fonts/PTSans/,
		UprightFont = *55F,
		BoldFont = *75F,
	]
	\newfontfamily\familynsf{PTN}[  % PT Sans Narrow Regular
		Path = ./fonts/PTSans/,
		UprightFont = *57F,
		BoldFont = *77F,
	]
	\usepackage{polyglossia}
	\setdefaultlanguage{english}
\else
	% compatibility with regular pdftex is desired, even if only for tests
	\usepackage{ucs}
	\usepackage[utf8x]{inputenc}
	\usepackage[T1]{fontenc}
	\let\familycaprm\rmfamily
	\let\familycapsf\sffamily
	\let\familycapnsf\sffamily
	\usepackage{paratype}
	\usepackage[main=english]{babel}
	% hacks for unavailable chars
	\DeclareTextCommandDefault{\textalpha}{\,\texttt{alpha}\,}
	\DeclareTextCommandDefault{\textbeta}{\,\texttt{beta}\,}
	\DeclareTextCommandDefault{\textlambda}{\,\texttt{lambda}\,}
	\DeclareTextCommandDefault{\textDelta}{\,\texttt{DELTA}\,}
	\DeclareUnicodeCharacter{9608}{\kern -0.1em\rule[-0.2em]{.8em}{1.2em}}
\fi

% font style
\let\emphasis\textit
\let\highlight\textbf

%pagesize
\usepackage{geometry}
\geometry{a4paper,total={105mm,243mm},left=35mm,top=18mm,marginparsep=7mm,marginparwidth=49mm,headsep=5.5mm,footskip=9mm}

%header & footer
\usepackage{fancyhdr}
\pagestyle{fancy}
\fancyhf{}
\fancyhead[LE]{\nouppercase\footnotesize\textsf{\color{gray25}BRT Planning Guide}}
\fancyhead[RO]{\nouppercase\footnotesize\textsf{\color{gray25}\leftmark}}
\fancyfoot[LE,RO]{\color{gray25}\thepage}
\renewcommand{\headrulewidth}{0pt}
\renewcommand{\chaptermark}[1]{\markboth{#1}{}}
\fancypagestyle{plain}{\fancyhf{}}

%colors
\usepackage{xcolor}
\definecolor{gray75}{gray}{0.25}
\definecolor{gray50}{gray}{0.5}
\definecolor{gray25}{gray}{0.75}
\definecolor{greenbrt}{cmyk}{0.75,0,0.75,0.25}
%volumecolors
\definecolor{volumeI}{cmyk}{0.2,1,0.4,0}
\definecolor{volumeII}{cmyk}{0.8,1,0.3,0.2}
\definecolor{volumeIII}{cmyk}{1,0.7,0.25,0.1}
\definecolor{volumeIV}{cmyk}{0.85,0.45,0.2,0}
\definecolor{volumeV}{cmyk}{0.65,0.2,1,0.05}
\definecolor{volumeVI}{cmyk}{0.25,0.6,1,0.1}
\definecolor{volumeVII}{cmyk}{0.25,1,1.4,0.15}
\definecolor{volumeVIII}{gray}{0.25}
\colorlet{volume}{greenbrt}  % used when there's no part number yet, such as in the temporary (and bodgy) cover

% headings
\usepackage{titlesec}
\setcounter{secnumdepth}{3}
\titleformat{\part}[display]{\bfseries\huge\color{volume\thepart}\familycapsf}{{\fontsize{10mm}{10mm}\selectfont VOLUME \thepart}}{0pt}{\vskip 5mm \rule{105mm}{1.5mm} \\ \vskip 2.25mm}[]
\titleformat{\chapter}[display]{\raggedright\sloppy\bfseries\LARGE\color{volume\thepart}\familycapsf}{}{0pt}{\thechapter.\ }[]
\titleformat*{\section}{\raggedright\sloppy\bfseries\Large\color{volume\thepart}\familycapsf}
\titleformat*{\subsection}{\raggedright\sloppy\bfseries\large\color{volume\thepart}\sffamily}
\titleformat*{\subsubsection}{\raggedright\sloppy\bfseries\sffamily}
\titlespacing*{\chapter}{0mm}{0mm}{2.25mm plus 2.25mm}[0pt]
\titlespacing*{\section}{0mm}{4.5mm plus 4.5mm}{2.25mm plus 2.25mm}[0pt]
\titlespacing*{\subsection}{0mm}{4.5mm plus 2.25mm}{2.25mm plus 2.25mm}[0pt]
\titlespacing*{\subsubsection}{0mm}{4.5mm plus 2.25mm}{2.25mm plus 2.25mm}[0pt]
\newcommand\setnextcounter[2]{\setcounter{#1}{#2}\addtocounter{#1}{-1}}
\newcommand\volume[2]{\setnextcounter{part}{#1}\part{#2}}
\let\uchapter=\chapter\renewcommand\chapter[2]{\setnextcounter{chapter}{#1}\uchapter{#2}}
\let\usection=\section\renewcommand\section[2]{\setnextcounter{section}{#1}\usection{#2}}
\let\usubsection=\subsection\renewcommand\subsection[2]{\setnextcounter{subsection}{#1}\usubsection{#2}}
\let\usubsubsection=\subsubsection\renewcommand\subsubsection[2]{\setnextcounter{subsubsection}{#1}\usubsubsection{#2}}

%workaround for titlesec 2.10.1 issue with missing labels
%http://tex.stackexchange.com/a/300259/46436
\makeatletter
\patchcmd{\ttlh@hang}{\parindent\z@}{\parindent\z@\leavevmode}{}{}
\patchcmd{\ttlh@hang}{\noindent}{}{}{}
\makeatother

% no-op macros
\newcommand\nop{}

%box
\usepackage{framed}

\renewenvironment{leftbar}[1][\hsize]
{%
    \def\FrameCommand
    {%
        {\hspace{-6.5mm} \color{volume\thepart}\vrule width 1mm}
        \hspace{3.0mm}
        \fboxsep=0mm
    }%
    \MakeFramed{\hsize#1}%
}
{\endMakeFramed}


\newcommand{\boxtitle}[2]{{\noindent\bfseries\color{volume\thepart}\familycapsf Box \thechapter.#1. #2} \\ }
\def\beginbox#1#2{ 
	\begin{leftbar} 
	\vskip 3.5mm
	\boxtitle{#1}{#2} 
	\let\manusmallfigure=\manumediumfigure \let\manusmallimgtable=\manumediumimgtable  % can't handle small figures or image tables for now, because marginpar isn't supported here
	\let\blockmargin=\nop\let\unblockmargin=\nop  % marginfix not supported here
}
\def\endbox{ \vskip 3.5mm \end{leftbar} }

% code
\newcommand\code[1]{\texttt{\color{gray50}#1}}  % TODO better isolate it from surrounding roman text
\def\begincode{
	\begingroup
	\vskip 3.5mm
	\parindent=0mm
	\parskip=0mm
	\baselineskip=0mm
	\color{gray50}
	\small
	\ttfamily  % FIXME
}
\def\endcode{
	\vskip 3.5mm
	\endgroup
}

%list
\usepackage{enumitem}
\setlist{nolistsep,leftmargin=7mm,rightmargin=7mm}
\labelindent=7mm

\usepackage{graphicx}
\usepackage{marginfix}  % improve marginpars, used for placing figures (and, in the future, tables) on the margin

% figures: \manu<size>figure{<path>}{<local number>}{caption}{copyright}
\newcommand\figcaption[3]{\footnotesize\sffamily\color{gray50}{\bfseries Figure \thechapter.#1.} #2 {\itshape #3}}
\newcommand{\manusmallfigure}[4]{
	\marginpar{
		\phantomsection
		\noindent\includegraphics[width=49mm]{#1} \\
		\noindent\RaggedRight\figcaption{#2}{#3}{#4}\par
	}
}
\newcommand{\manumediumfigure}[4]{
	{
		\phantomsection
		\vskip 2.25mm plus 2.25mm
		\noindent\includegraphics[width=105mm]{#1}
		\vskip 2.25mm
		\noindent\RaggedRight\figcaption{#2}{#3}{#4}
		\vskip 2.25mm plus 2.25mm
	}
}
\newcommand{\manulargefigure}[4]{
	\blockmargin
	{
		\phantomsection
		\vskip 2.25mm plus 2.25mm
		\noindent\includegraphics[width=161mm]{#1}
		\vskip 2.25mm
		\vbox{\hsize=161mm\noindent\RaggedRight\figcaption{#2}{#3}{#4}}
		\vskip 2.25mm plus 2.25mm
	}
	\unblockmargin
}

% img tables: \manu<size>imgtable{<path>}{<local number>}{<title>}
\newcommand\tabletitletext[2]{{\bfseries\color{volume\thepart}\sffamily Table \thechapter.#1. #2}}
\newcommand{\manusmallimgtable}[4]{
	\marginpar{
		\phantomsection
		\noindent\RaggedRight\footnotesize\tabletitletext{#2}{#3}\par
		\noindent\includegraphics[width=49mm]{#1}
	}
}
\newcommand{\manumediumimgtable}[4]{
	{
		\phantomsection
		\vskip 2.25mm plus 2.25mm
		\noindent\RaggedRight\tabletitletext{#2}{#3}
		\vskip 2.25mm plus 2.25mm
		\noindent\includegraphics[width=105mm]{#1}
		\vskip 2.25mm plus 2.25mm
	}
}
\newcommand{\manulargeimgtable}[4]{
	\blockmargin
	{
		\phantomsection
		\vskip 2.25mm plus 2.25mm
		\vbox{\hsize=161mm\noindent\RaggedRight\tabletitletext{#2}{#3}}
		\vskip 2.25mm plus 2.25mm
		\noindent\includegraphics[width=161mm]{#1}
		\vskip 2.25mm plus 2.25mm
	}
	\unblockmargin
}

% non sectioning titles: \manutitle{<name>}
\newcommand{\manutitle}[1]{
	{
	\setlength{\parindent}{0cm}
	\vskip 2.25mm plus 2.25mm
	{\bfseries\sffamily #1}
	\vskip 2.25mm plus 2.25mm
	}
}

% unsupported things in xetex
\ifxetex
	\let\manusmallfigure=\nop \let\manumediumfigure=\nop \let\manulargefigure=\nop
	\let\manusmallimgtable=\nop \let\manumediumimgtable=\nop \let\manulargeimgtable=\nop
\fi

%quotation {quote}{name, about, 1900 - 2000}
\renewcommand{\quotation}[2]{\begingroup\leftskip=14mm\rightskip=14mm {\small\color{gray50} \noindent\textit{``#1''} \\\hbox{}\hfill --- #2} \par\endgroup \vskip 2.25mm plus 2.25mm }

%tables
%note: \tablemodule*46*2 must match \textwidth+\marginparsep+\marginparwidth
\usepackage{ifoddpage}
\newdimen\tablemodule \tablemodule=\textwidth \divide\tablemodule by 60
\newcommand{\tabletitle}[2]{{\phantomsection\vskip 2.25mm plus 2.25mm\noindent\tabletitletext{#1}{#2}} \vskip 2.25mm plus 2.25mm }
\newcommand{\tablesource}[1]{\noindent\RaggedRight\footnotesize\sffamily\color{gray50}{\bfseries Source:} #1 \par \smallskip}

\AtBeginDocument{
	\parskip=0mm plus 4.5mm
	\baselineskip=4.5mm
	\parindent=7mm
	\color{gray75}
	% can't do any typesetting here, still part of the preamble
}
\AfterEndPreamble{
	\part*{The BRT Planning Guide}
}

% math
\usepackage{amsmath}

%tmp
\usepackage{lipsum}

\usepackage[
    bookmarks,
    bookmarksnumbered,
    colorlinks,
    draft=false,
    pdftitle={The BRT Planning Guide},
    pdfauthor={Institute for Transportation and Development Policy},
    unicode,
    allcolors=greenbrt,
    breaklinks=true
]{hyperref}  % TODO reuse infos

% url
\renewcommand*{\UrlFont}{\rmfamily\relax}
% allow urls to break in more places than the default behaviour
\expandafter\def\expandafter\UrlBreaks\expandafter{\UrlBreaks
  \do\a\do\b\do\c\do\d\do\e\do\f\do\g\do\h\do\i\do\j
  \do\k\do\l\do\m\do\n\do\o\do\p\do\q\do\r\do\s\do\t
  \do\u\do\v\do\w\do\x\do\y\do\z\do\A\do\B\do\C\do\D
  \do\E\do\F\do\G\do\H\do\I\do\J\do\K\do\L\do\M\do\N
  \do\O\do\P\do\Q\do\R\do\S\do\T\do\U\do\V\do\W\do\X
  \do\Y\do\Z}

% better bookmarks than with just hyperref (replaces them)
\usepackage{bookmark}

% make hyphenation effective within \raggedright text
\usepackage{ragged2e}
